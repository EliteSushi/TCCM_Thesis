% !TeX root = ../../thesis.tex
\chapter{Theoretical Background}\label{ch:theory}

\section{Self Consistent Field Methods} \label{sec:SCF}
The Hartree-Fock (HF) method stands as a foundational approximation in the realm of electronic structure theory. Its primary objective is to provide an approximate solution to the intricate many-electron time-independent Schrödinger equation, which governs the behavior of electrons within atoms and molecules within the Born-Openhaimer approximation:\\
\begin{equation}\label{eq:TISE}
    \hat{H}_e \Psi_e = E_e \Psi_e
\end{equation}
The HF method achieves this by employing a significant simplification: it rests on the assumption that each electron moves independently within an average electrostatic field generated by all the other electrons in the system. This conceptual framework is often referred to as the mean-field approximation.\\
The cornerstone of the HF approximation is the use of a Slater determinant to represent the \textit{N}-electron wavefunction. A Slater determinant is a specific mathematical expression formed by taking the antisymmetrized product of N individual one-electron wavefunctions, known as spin-orbitals ($\chi$):\\

\begin{equation}\label{eq:SlaterDet}
    \Psi(\mathbf{r}_1, \mathbf{r}_2, \dots, \mathbf{r}_N) = \frac{1}{\sqrt{N!}}
    \begin{vmatrix}
      \chi_1(\mathbf{r}_1) & \chi_2(\mathbf{r}_1) & \cdots & \chi_N(\mathbf{r}_1) \\
      \chi_1(\mathbf{r}_2) & \chi_2(\mathbf{r}_2) & \cdots & \chi_N(\mathbf{r}_2) \\
      \vdots & \vdots & \ddots & \vdots \\
      \chi_1(\mathbf{r}_N) & \chi_2(\mathbf{r}_N) & \cdots & \chi_N(\mathbf{r}_N)
    \end{vmatrix}
  \end{equation}

The determinantal form is crucial as it inherently satisfies both the Pauli exclusion principle, and the antisymmetry requirement of fermions. Mathematically, the Hartree-Fock method is rooted in the variational principle, which states that any approximate wavefunction will yield an energy greater than or equal to the true ground-state energy. The energy expectation value for a Slater determinant according to HF is:

\begin{equation}\label{EHF}
    \begin{aligned}
        E_{HF} &= \sum_{i=1}^{N} \hat{F}_i \Psi \\
            &= \sum_{i=1}^{N} \hat{h}(i) + \sum_{i,j=1}^{N} (\hat{J}_j(i) - \hat{K}_j(i)) \\ 
            &= \sum_{i=1}^{N} \langle \chi_i | \hat{h} | \chi_i \rangle + \frac{1}{2} \sum_{i,j=1}^{N} \langle \chi_i \chi_j || \chi_i \chi_j \rangle
    \end{aligned}
\end{equation}
\\
The Fock operator, $\hat{F}$, is made up from $\hat{h}$, the one-electron core Hamiltonian operator (kinetic energy and electron-nucleus attraction); $\hat{J}$, the Coulomb operator, describing the electrostatic repulsion between electron i and the average charge distribution of electron j, and $\hat{K}^j(i)$ is the exchange operator, a purely quantum mechanical term arising from the antisymmetry requirement. The Hartree-Fock equations are then expressed as an eigenvalue problem for the Fock operator:
$ \hat{F}(i) \chi_i(i) = \epsilon_i \chi_i(i) $
where $\epsilon$ are the orbital energies corresponding to the spin-orbitals $\chi$.\\

The Hartree-Fock equations are inherently non-linear because the Fock operator depends on the wavefunctions of all the other electrons, their interactions are coupled. Consequently, these equations cannot be solved analitically and necessitate an iterative procedure known as the Self-Consistent Field (SCF) method. The SCF procedure typically involves the following steps: An initial guess for the molecular orbitals (or spin-orbitals) is made. Using this initial guess, the Fock operator, an effective one-electron Hamiltonian, is constructed. The Fock operator includes terms for the kinetic energy of the electrons, their attraction to the nuclei, the average repulsion from all other electrons (Coulomb interaction), and the exchange interaction arising from the antisymmetry of the wavefunction. The Hartree-Fock equations (or the Roothaan equations when a basis set is used) are then solved by diagonalizing the Fock operator to obtain a new set of molecular orbitals and their corresponding energies (orbital energies). This new set of orbitals is compared to the previous set. If the change is below a predefined threshold (convergence criteria), the procedure is considered converged, and the self-consistent field is achieved. If convergence is not reached, the process is repeated.

\subsection{The Electron Correlation Problem}
\label{subsec:electron_correlation}
Despite its fundamental importance and reasonable accuracy for many molecular properties, the Hartree-Fock (HF) method is inherently limited by its neglect of the instantaneous correlation between the motions of electrons . In the HF approximation, each electron is treated as moving independently within a static, average field created by all other electrons . This mean-field approach fails to account for the fact that electrons, being negatively charged, will instantaneously repel each other, leading to a dynamic correlation in their movements as they try to avoid each other in space.\\
The primary consequence of neglecting electron correlation in the HF approximation is an overestimation of the electron-electron repulsion energy. While the HF method does account for the exchange interaction exactly (Fermi correlation), it completely neglects the Coulomb, or dynamic, correlation, which describes the instantaneous spatial correlation of electrons due to their Coulombic repulsion. This omission is a fundamental limitation of the HF method and is responsible for its inability to accurately predict certain phenomena, such as London dispersion forces.\\
The correlation energy ($E_{corr}$) is defined as the difference between the exact non-relativistic energy ($E_{exact}$) of the system and the energy obtained in the Hartree-Fock limit ($E_{HF\,limit}$) for the same basis set:

\begin{equation}\label{eq:Ecorr}
    E_{corr} =E_{exact}-E_{HF\,limit}
\end{equation}

The Hartree-Fock limit represents the theoretical energy that would be obtained if the HF equations were solved using a complete basis set, which is practically unattainable but serves as a theoretical benchmark. In practical calculations with finite basis sets, the correlation energy is often considered relative to the HF energy obtained with that specific basis. The correlation energy is always negative because the exact energy is lower than the Hartree-Fock energy due to the variational principle. For systems with multiple bonds or during bond breaking, methods that go beyond the HF approximation are necessary to achieve accurate results.
\subsection{Brief Introduction to Correlated Methods}
\label{subsec:correlated_methods}
Correlated methods aim to include the effects of the instantaneous interactions between electrons that are neglected in the mean-field approximation of HF theory.
\subsubsection{Møller-Plesset Perturbation Theory}
Møller-Plesset (MP) perturbation theory offers a systematic way to improve upon the HF energy by treating the electron correlation as a perturbation to the Hartree-Fock Hamiltonian.The Fock operator is taken as the zeroth-order Hamiltonian, and the difference between the exact electron-electron repulsion and the Fock operator is considered the perturbation. The energy and wavefunction are then expanded as a series in terms of the perturbation strength. The first-order energy correction in MP theory is zero, so the first non-trivial correction to the HF energy appears at the second order, giving rise to the MP2 method. The MP2 energy correction for a closed-shell molecule is given by:
\begin{equation} \label{eq:MP2}
    E_{MP2} = - \frac{1}{4} \sum_{ij}^{occ} \sum_{ab}^{virt} \frac{|\langle i j || a b \rangle|^2}{\epsilon_a + \epsilon_b - \epsilon_i - \epsilon_j}
\end{equation}

Where $i,j$ denote occupied molecular orbitals, $a,b$ denote virtual molecular orbitals, $\epsilon$ are the corresponding orbital energies from the HF calculation, and $\langle ij||ab \rangle$ are the antisymmetrized two-electron integrals in the molecular orbital basis. MP theory can be extended to higher orders (MP3, MP4, etc.) to achieve greater accuracy, although the computational cost increases significantly with each order.
\subsubsection{Density Functional Theory}
Density Functional Theory (DFT) provides an alternative approach to incorporating electron correlation by focusing on the electron density of the system rather than the wavefunction, reducing the degrees of freedom of the system from $3N-3$ to just $3$. The fundamental principle of DFT is that the ground state energy of a system is a unique functional of its electron density. The Kohn-Sham formulation of DFT introduces a fictitious system of non-interacting electrons that have the same density as the real interacting system. The energy of this system is expressed as a functional of the density. The Kohn-Sham equations, which are solved iteratively to obtain the electron density, are given by:

\begin{equation}\label{eq:KSDFT}
    \left( -\frac{1}{2} \nabla^2 + V_{ext}(\mathbf{r}) + V_H(\mathbf{r}) + V_{XC}[\rho(\mathbf{r})] \right) \psi_i(\mathbf{r}) = \epsilon_i \psi_i(\mathbf{r})
\end{equation}

Where $V_ext$ respresents the external potential, $V_H(\mathbf{r}) = \int \frac{\rho(\mathbf{r}')}{|\mathbf{r} - \mathbf{r}'|} d\mathbf{r}'$ is the hartree potential, $V_{XC}$ is the Exchange-Correlation potential and $\rho(\mathbf{r})$ is the electron density. The exchange-correlation functional is the most challenging part of DFT, as it is not known exactly and must be approximated. The accuracy of DFT calculations depends heavily on the choice of exchange-correlation functional.

\subsubsection{Configuration Interaction}
Configuration Interaction (CI) methods improve upon HF by expressing the electronic wavefunction as a linear combination of the HF ground state determinant and excited state determinants:
\begin{equation} \label{eq:CI}
     |\Psi_{CI} \rangle = c_0 |\Phi_0 \rangle + \sum_{ia} c_{ia} |\Phi_{ia} \rangle + \sum_{ijab} c_{ijab} |\Phi_{ijab} \rangle + \dots
\end{equation}
These excited determinants correspond to the promotion of electrons from occupied to virtual orbitals. The coefficients of this linear combination are determined by the variational principle to minimize the energy.
where $|\Phi_0 \rangle$ is the HF ground state determinant, $|\Phi_{ia} \rangle$ are singly excited determinants, $|\Phi_{ijab} \rangle$ are doubly excited determinants, and c are the CI coefficients. Full CI, which includes all possible excited determinants, is exact within the basis set but computationally prohibitive for all but the simplest systems. Truncated CI methods, such as CIS (singles) and CISD (singles and doubles), are more practical but lack size extensivity.

\section{Coupled Cluster Theory} \label{sec:CCTheory}
The coupled cluster (CC) theory is considered the gold-standard method in quantum chemistry. Similarly to CI, the CC method expands the wavefunction as a liinear combination of Slater determinats. However, the CC method results into a size-extensive and size-consistent wavefunction by using an exponential ansatz.

\begin{equation}\label{CCWavefunc}
    | \Psi_{CC} \rangle = e^{\hat{T}} | \Psi_{0} \rangle
\end{equation}
\\Where $| \Psi_{0} \rangle$ is the HF ground state, and $\hat{T}$ is the cluster operator. The cluster operator $\hat{T}$ is the central component of CC theory and is defined as a sum of excitation operators:
\begin{equation}
    \hat{T} = \hat{T}_1 + \hat{T}_2 + \hat{T}_3 + \dots + \hat{T}_N
\end{equation}
where $N$ is the total number of electrons in the system. Each term in this sum corresponds to a specific level of excitation:
\begin{itemize}
    \item $\hat{T}_1 = \sum_{i}^{\text{occ}} \sum_{a}^{\text{virt}} t_i^a a_a^{\dagger} a_i$ represents single excitations.
    \item $\hat{T}_2 = \frac{1}{4} \sum_{i,j}^{\text{occ}} \sum_{a,b}^{\text{virt}} t_{ij}^{ab} a_a^{\dagger} a_b^{\dagger} a_j a_i$ represents double excitations.
    \item Higher-order excitation operators $\hat{T}_3, \hat{T}_4, \dots$ describe simultaneous excitation of three, four, and more electrons.
\end{itemize}
The coefficients $t_i^a$, $t_{ij}^{ab}$, etc., are cluster amplitudes to determined by solving the coupled cluster Schr\"{o}dinger equation.

\iffalse One of the most significant advantages of coupled cluster theory is its property of size consistency. A size-consistent method correctly describes the energy of a system composed of multiple non-interacting subsystems as the sum of the energies of the individual subsystems. The exponential form, when expanded as a Taylor series,
\begin{equation}
    e^{\hat{T}} = 1 + \hat{T} + \frac{1}{2!} \hat{T}^2 + \dots
\end{equation}
inherently includes terms that represent disconnected clusters, which are essential for size consistency. \fi
The energy of the system is obtained by projecting the Schrödinger equation onto the Hartree-Fock reference determinant:
\begin{equation}\label{eq:CCEnergy}
    E_{CC}=\langle \Psi_{0} | e^{-\hat{T}} \hat{H} e^{\hat{T}} | \Psi_{0} \rangle
\end{equation}
Using the Baker-Campbell-Hausdorff (BCH) expansion, the exponential operators in Eq. \ref{eq:CCEnergy} can be simplified to a series of commutators which ends at the fourth order. The cluster operator $\hat{T}$ can be truncated at different levels of excitation:
\begin{itemize}
    \item \textbf{CCD (Coupled Cluster Doubles)}: This is the simplest approximation in the CC family, where the cluster operator is truncated to include only single excitations: $\hat{T} \approx \hat{T}_2$. Due to the Brilluin's theorem, the amplitudes of single excitations are 0. 
    \item \textbf{CCSD (Coupled Cluster Singles and Doubles)}: This is one of the most widely used and generally accurate coupled cluster methods, where the cluster operator includes both single and double excitations:$\hat{T} \approx \hat{T}_1 + \hat{T}_2$.
    \item \textbf{CCSDT (Coupled Cluster Singles, Doubles, and Triples)}: $\hat{T} \approx \hat{T}_1 + \hat{T}_2 + \hat{T}_3$.
    The hierarchy can be extended to include even higher levels of excitation, converging to the Full Configuration Interaction (Full CI) limit. Full CI includes all possible excitations within a given one-electron basis set and represents the exact solution to the non-relativistic Schrödinger equation in that basis.
\end{itemize}

\begin{table}[h!]
    \centering
    \begin{tabular}{ccc}
        Method & Operation count & Memory \\
        \hline
        HF & $O(N^4)$ & $O(N^4)$ \\
        MP2 & $O(N^5)$ & $O(N^4)$ \\
        CCD/CCSD & $O(N^6)$ & $O(N^4)$ \\
        CC2 & $O(N^{5})$ & $O(N^4)$ \\
    \end{tabular}
    \caption{Computational Scaling of quantum chemistry Methods.}
    \label{tab:qc_scaling}
\end{table}


\section{Second Approximate Coupled Cluster}\label{sec:CC2Theory}
Second Approximate Coupled Cluster (CC2) belongs to the broader family of CCn approximate coupled cluster methods, where the 'n' in CCn indicates the level of approximation within a perturbative hierarchy. These methods aim to reduce the computational cost associated with standard CC truncations while still retaining a reasonable level of accuracy.\\

In CC2 the equations for the single amplitudes ($t^a_i$) are the same as those in CCSD, while the doubles amplitudes ($t^ab_ij$) are calculated using the non-iterative expression for MP2 (Eq \ref{eq:MP2}). The resulting expression for the CC2 correlation energy is:
\begin{equation}\label{CC2Energy}
    E_{CC2} = \sum_{ij}^{occ} \sum_{ab}^{virt} \frac{1}{4}\frac{|\langle i j || a b \rangle|^2}{\epsilon_a + \epsilon_b - \epsilon_i - \epsilon_j}  + \sum_{i}^{occ} \sum_{a}^{virt} \hat{F}_{ai} t^a_i 
\end{equation}

The perturbative treatment of the doubles amplitudes in CC2, reduces the computational cost compared to CCSD, Table \ref{tab:qc_scaling}. While this approximation can lead to a less accurate description of electron correlation, the inclusion of zeroth-order singles amplitudes allows for an approximate description of orbital relaxation, which often leads to higher quality results compared to MP2.

\section{Equation-of-Motion Methods} \label{sec:eom_theory}
Equation-of-Motion Coupled Cluster (EOM-CC) methods are an extension of ground-state coupled cluster theory, which provide a versatile and accurate framework for calculating a variety of excited (EE), ionized (IP) and electron-attached (EA) states. In the EOM-CC, the target electronic state is described by applying a linear excitation operator $\hat{R}$ to a reference state, which typically is the coupled cluster wavefunction of the ground state. The form of the operator $\hat{R}$ is similar to the cluster operator and chosen to access the desired type of target state. 
\iffalse For instance, for EOM-EA, $\hat{R}$ is:
\begin{equation} \label{REA}
    \hat{R}_{EA} = R_1 + R_2 + ... = \sum_{ai} r^a_i a^{\dagger}_a a_i + \frac{1}{4}\sum_{abij} r^a_i a^{\dagger}_a a_i a^{\dagger}_b a_j + ...  
\end{equation} \fi
The target state wavefunction can then be expressed as $|\Psi_{\text{EOM}}\rangle = \hat{R} |\Psi_{\text{CC}}\rangle = \hat{R} e^{\hat{T}} |\Phi_{\text{HF}}\rangle$. \\

%%%%REVISE FROM HERE%%%%%
MAKE FIGURE OF FLAVORS OF EOM\\
Each of the flavors of EOM-CC involves a tailored form of the excitation operator \^{R} and is applied to a specific reference state to target the desired electronic states.
The Equation-of-Motion Electron Attachment (EOM-EA) method is a specific variant of EOM-CC theory that is designed to calculate the electronic states of systems with one more electron than the reference state, which is typically the ground state of the neutral molecule. In the EOM-EA formalism, the wavefunction for the $k^{\text{th}}$ electron-attached state ($|\Psi_k^{N+1}\rangle$) is obtained by applying a linear electron attachment operator $R_k^{N+1}$ to the coupled cluster wavefunction of the $N$-electron reference state ($|\Psi_0\rangle$):

\begin{equation}
    |\Psi_k^{N+1}\rangle = R_k^{N+1} |\Psi_0\rangle.
\end{equation}

The electron attachment operator $R_k^{N+1}$ typically includes terms that describe the addition of one electron to an unoccupied orbital (a one-particle creation operator, $R_1$) and terms that describe the addition of one electron accompanied by the excitation of another electron from an occupied to an unoccupied orbital (a two-particle and one-hole creation operator, $R_2$):

\begin{equation}
    R_k^{N+1} = R_1^k + R_2^k = \sum_{a} r_a^k a_a^{\dagger} + \sum_{b<a} \sum_{j} r_{ba}^{j} a_b^{\dagger} a_a^{\dagger} a_j,
\end{equation}

where $a$ and $b$ denote virtual orbitals, $j$ denotes an occupied orbital, and $r_a^k$ and $r_{ba}^{j}$ are the coefficients (amplitudes) to be determined. The electron affinities (EAs) of the system are then obtained as the negative of the eigenvalues ($\Delta E_k$) by solving the equation of motion for the $R_k^{N+1}$ operator with respect to the similarity-transformed Hamiltonian:

\begin{equation}
    \bar{H}^{N} = e^{-T} H e^{T} - \langle \Phi_0 | e^{-T} H e^{T} | \Phi_0 \rangle,
\end{equation}

where $T$ is the ground-state coupled cluster excitation operator.

EOM-EA is a valuable tool in quantum chemistry for accurately calculating vertical electron affinities of atoms and molecules, which represent the energy change when an electron is added to the neutral species at a fixed geometry. This method is particularly useful for studying the electronic structure and stability of anions, including various types such as dipole-bound anions (where a weakly bound electron exists due to the dipole moment of the neutral molecule) and radical anions. EOM-EA can also be applied to investigate electron attachment processes in chemical reactions and to model photoelectron spectra arising from the detachment of electrons from anionic species. Its ability to effectively account for electron correlation makes it a robust approach for systems where simpler methods, like Hartree-Fock or Koopmans' theorem, may not provide accurate electron affinity values.

\subsection{Dyson orbitals}

Dyson orbitals, also known as generalized overlap amplitudes or transition orbitals, offer a powerful framework for understanding the changes in the electronic structure of a quantum system upon the removal or addition of an electron. Formally, a Dyson orbital is defined as the quantum mechanical overlap between the wavefunction of an initial $N$-electron state ($|\Psi_I^N\rangle$) and the wavefunction of the final state with $N-1$ electrons ($|\Psi_F^{N-1}\rangle$) in the case of electron detachment (ionization), or with $N+1$ electrons ($|\Psi_F^{N+1}\rangle$) in the case of electron attachment.  

For the process of electron detachment from an initial $N$-electron state $I$ to a final $N-1$ electron state $F$, the Dyson orbital $\phi_{Dyson}^{FI}(r_1)$ is mathematically defined as:
\begin{equation}
    \phi_{Dyson}^{FI}(r_1) = \sqrt{N} \int \Psi_F^{N-1*}(r_2,\dots,r_N) \Psi_I^N(r_1, r_2,\dots,r_N)\,dr_2 \dots dr_N
\end{equation}
where $r_i$ represents the combined space and spin coordinates of the $i$-th electron, and the integration is carried out over the coordinates of all electrons except one. The factor of $\sqrt{N}$ is a normalization convention. 

Similarly, for the process of electron attachment from an initial $N$-electron state $I$ to form a final $N+1$ electron state $F$, the Dyson orbital $\phi_{Dyson}^{IF}(r_1)$ is defined as:
\begin{equation}
    \phi_{Dyson}^{IF}(r_1) = \sqrt{N+1} \int \Psi_I^{N*}(r_2,\dots,r_{N+1}) \Psi_F^{N+1}(r_1, r_2,\dots,r_{N+1})\,dr_2 \dots dr_{N+1}
\end{equation}
Here, the integration is over the coordinates of all electrons except one, and the factor of $\sqrt{N+1}$ is the normalization factor.

The fundamental definitions of Dyson orbitals for electron detachment and attachment have been provided in the preceding section. In practice, particularly when employing Equation-of-Motion Coupled Cluster (EOM-CC) theory, Dyson orbitals can be conveniently expressed as a linear combination of the molecular orbitals ($\phi_p(r)$) of the neutral system:
\begin{equation}
    \phi_{Dyson}(r) = \sum_p \gamma_p \phi_p(r)
\end{equation}
where $\gamma_p$ are the coefficients that quantify the contribution of each molecular orbital to the Dyson orbital. These coefficients are directly related to the transition amplitudes obtained from EOM-IP calculations for ionization or EOM-EA calculations for electron attachment. 

For ionization from an $N$-electron state to an $N-1$ electron state, the coefficient $\gamma_p$ is given by:
\begin{equation}
    \gamma_p = \langle \Psi_F^{N-1} | a_p | \Psi_I^N \rangle
\end{equation}
where $a_p$ is the annihilation operator for the $p$-th molecular orbital. 

For electron attachment from an $N$-electron state to an $N+1$ electron state, the coefficient is:
\begin{equation}
    \gamma_p = \langle \Psi_F^{N+1} | a_p^\dagger | \Psi_I^N \rangle
\end{equation}
where $a_p^\dagger$ is the creation operator.

The norm squared of the Dyson orbital, denoted as the pole strength or probability factor ($P$), is a crucial quantity related to the intensity of the spectroscopic transition. It is calculated by integrating the squared modulus of the Dyson orbital over all space:
\begin{equation}
    P = \int |\phi_{Dyson}(r)|^2 \,dr = \sum_{p,q} \gamma_p^* \gamma_q \langle \phi_p | \phi_q \rangle
\end{equation}
If the molecular orbitals form an orthonormal basis set, this expression simplifies to:
\begin{equation}
    P = \sum_p |\gamma_p|^2
\end{equation}
The pole strength ranges from 0 to 1 and provides a direct measure of the one-electron character of the ionization or electron attachment process.

Physically, a Dyson orbital can be interpreted as representing the correlated state of the electron that is either removed from the initial $N$-electron system or added to it to form the final $N\pm1$ electron system. In the context of ionization or photodetachment processes, the Dyson orbital can be thought of as the wavefunction of the ejected electron, providing valuable information about its spatial distribution and momentum characteristics. The associated pole strength quantifies the probability of finding the resulting ion in a particular final electronic state after the electron has been removed or attached. 

Dyson orbitals have found a wide array of applications in both theoretical chemistry and spectroscopy:
\begin{itemize}
    \item \textbf{Photoelectron Spectroscopy (PES):} Dyson orbitals are indispensable for the interpretation and prediction of photoelectron spectra. The energies associated with the Dyson orbitals are related to the ionization energies observed in PES, and the pole strengths provide crucial information about the relative intensities of the spectral peaks.
    \item \textbf{Electron Momentum Spectroscopy (EMS):} EMS is an experimental technique that directly probes the momentum-space electron density of molecules. The experimental data obtained from EMS can be directly compared with the Fourier transforms (momentum-space representations) of theoretically calculated Dyson orbitals, providing a powerful means of validating electronic structure calculations.
    \item \textbf{Other Spectroscopic Techniques:} The utility of Dyson orbitals extends to the interpretation of other spectroscopic methods that involve ionization or electron attachment, including Compton scattering, Penning ionization electron spectroscopy, and angle-resolved photoelectron spectroscopy (ARPES).
    \item \textbf{Theoretical Analysis of Electronic Structure and Bonding:} Beyond spectroscopy, Dyson orbitals serve as valuable tools for analyzing electronic structure and bonding. They can be used to visualize and quantify the changes in electron density that occur upon ionization or electron attachment, providing a deeper understanding of chemical reactivity. Furthermore, they can be employed to generalize concepts such as frontier molecular orbitals to incorporate electron correlation effects, offering a more accurate picture of molecular reactivity.
\end{itemize}

%%%%%%%%%%%%%%%%%%%%%%%%%%%%%%%%%%%%%%%%%%%%%%%%%%
% Keep the following \cleardoublepage at the end of this file, 
% otherwise \includeonly includes empty pages.
\cleardoublepage

% vim: tw=70 nocindent expandtab foldmethod=marker foldmarker={{{}{,}{}}}