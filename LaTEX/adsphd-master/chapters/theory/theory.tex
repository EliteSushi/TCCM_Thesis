% !TeX root = ../../thesis.tex
\chapter{Theoretical Background}\label{ch:theory}

\section{Self Consistent Field Methods} \label{sec:SCF}
The Hartree-Fock (HF) method stands as a method in electronic structure theory. Its primary objective is to provide an approximate solution to the many-electron time-independent Schrödinger equation within the Born-Openhaimer approximation, which governs the behavior of electrons within atoms and molecules:
\begin{equation}\label{eq:TISE}
    \hat{H}_e \Psi_e = E_e \Psi_e
\end{equation}
The HF method achieves this by assuming that each electron moves independently within an average electrostatic field generated by all the other electrons in the system. In the HF method the \textit{N}-electron wavefunctionis is represented by a Slater determinant, which is formed by taking the antisymmetrized product of N individual one-electron wavefunctions, the spin-orbitals ($\chi$):
\begin{equation}\label{eq:SlaterDet}
    \Psi(\mathbf{r}_1, \mathbf{r}_2, \dots, \mathbf{r}_N) = \frac{1}{\sqrt{N!}}
    \begin{vmatrix}
      \chi_1(\mathbf{r}_1) & \chi_2(\mathbf{r}_1) & \cdots & \chi_N(\mathbf{r}_1) \\
      \chi_1(\mathbf{r}_2) & \chi_2(\mathbf{r}_2) & \cdots & \chi_N(\mathbf{r}_2) \\
      \vdots & \vdots & \ddots & \vdots \\
      \chi_1(\mathbf{r}_N) & \chi_2(\mathbf{r}_N) & \cdots & \chi_N(\mathbf{r}_N)
    \end{vmatrix}
  \end{equation}
The determinantal inherently satisfies both the Pauli exclusion principle, and the antisymmetry requirement of fermions. The energy expectation for a Slater determinant according to HF is variational and can be computed as:
\begin{equation}\label{EHF}
    \begin{aligned}
        E_{HF} &= \sum_{i=1}^{N} \hat{F}_i \Psi \\
            &= \sum_{i=1}^{N} \hat{h}(i) + \sum_{i,j=1}^{N} (\hat{J}_j(i) - \hat{K}_j(i)) \\ 
            &= \sum_{i=1}^{N} \langle \chi_i | \hat{h} | \chi_i \rangle + \frac{1}{2} \sum_{i,j=1}^{N} \langle \chi_i \chi_j || \chi_i \chi_j \rangle
    \end{aligned}
\end{equation}
Where, $\hat{F}$ is the Fock operator. $\hat{F}$ is made up from $\hat{h}$, the one-electron core Hamiltonian operator (kinetic energy and electron-nucleus attraction); $\hat{J}_j(i)$, the Coulomb operator, describing the electrostatic repulsion between electron i and the average charge distribution of electron j, and $\hat{K}_j(i)$ is the exchange operator, a purely quantum mechanical term arising from the antisymmetry principle. \\
The Hartree-Fock equations are inherently non-linear because the Fock operator depends on the wavefunctions of all the other electrons, their interactions are coupled. Consequently, these equations cannot be solved analitically and necessitate an iterative procedure known as the Self-Consistent Field (SCF) method.

\subsection{Electron Correlation}
\label{subsec:electron_correlation}
The Hartree-Fock (HF) method is inherently limited by its neglect of the instantaneous correlation between the motions of electrons. In the HF approximation, each electron is treated as moving independently within a static, average field created by all other electrons. This mean-field approach fails to account for the fact that electrons, being negatively charged, will instantaneously repel each other, leading to a dynamic correlation in their movements as they try to avoid each other in space.\\
This neglection leads to an overestimation of the electron-electron repulsion energy, and is responsible for its inability to accurately predict certain phenomena, such as London dispersion forces. The difference between the exact non-relativistic energy of the system and the energy obtained in the Hartree-Fock limit is defined as the correlation energy and is always negative due to the variational principle. Correlated methods aim to include the effects of the instantaneous interactions between electrons that are neglected in the mean-field approximation of HF theory. Several correlated methods used during this work are:

\subsubsection{Møller-Plesset Perturbation Theory}
Møller-Plesset (MP) perturbation theory offers a systematic way to improve upon the HF energy by treating the electron correlation as a perturbation to the Hartree-Fock Hamiltonian.The Fock operator is taken as the zeroth-order Hamiltonian, and the difference between the exact electron-electron repulsion and the Fock operator is considered the perturbation. The energy and wavefunction are then expanded as a series in terms of the perturbation strength. The first-order energy correction in MP theory is zero, so the first non-trivial correction to the HF energy appears at the second order, giving rise to the MP2 method. The MP2 energy correction for a closed-shell molecule is given by:
\begin{equation} \label{eq:MP2}
    E_{MP2} = - \frac{1}{4} \sum_{ij}^{occ} \sum_{ab}^{virt} \frac{|\langle i j || a b \rangle|^2}{\epsilon_a + \epsilon_b - \epsilon_i - \epsilon_j}
\end{equation}
Where $i,j$ denote occupied molecular orbitals, $a,b$ denote virtual molecular orbitals, and $\epsilon$ are the corresponding orbital energies from the HF calculation. MP theory can be extended to higher orders (MP3, MP4, etc.) to achieve greater accuracy, although the computational cost increases significantly with each order.

\subsubsection{Density Functional Theory}
Density Functional Theory (DFT) provides an alternative approach to incorporating electron correlation by focusing on the electron density of the system rather than the wavefunction, reducing the degrees of freedom of the system from $3N-3$ to just $3$. The fundamental principle of DFT is that the ground state energy of a system is a unique functional of its electron density:
\begin{equation}\label{eq:KSDFT}
    \left( -\frac{1}{2} \nabla^2 + V_{ext}(\mathbf{r}) + V_H(\mathbf{r}) + V_{XC}[\rho(\mathbf{r})] \right) \psi_i(\mathbf{r}) = \epsilon_i \psi_i(\mathbf{r})
\end{equation}
Where $V_ext$ respresents the external potential, $V_H(\mathbf{r}) = \int \frac{\rho(\mathbf{r}')}{|\mathbf{r} - \mathbf{r}'|} d\mathbf{r}'$ is the hartree potential, $V_{XC}$ is the Exchange-Correlation potential and $\rho(\mathbf{r})$ is the electron density. The exchange-correlation functional is the most challenging part of DFT, as it is not known exactly and must be approximated. The accuracy of DFT calculations depends heavily on the choice of exchange-correlation functional.

\subsubsection{Configuration Interaction}
Configuration Interaction (CI) methods improve upon HF by expressing the electronic wavefunction as a linear combination of the HF ground state determinant and excited state determinants:
\begin{equation} \label{eq:CI}
     |\Psi_{CI} \rangle = c_0 |\Phi_0 \rangle + \sum_{ia} c_{ia} |\Phi_{ia} \rangle + \sum_{ijab} c_{ijab} |\Phi_{ijab} \rangle + \dots
\end{equation}
Where $|\Phi_0 \rangle$ is the HF ground state determinant, $|\Phi_{ia} \rangle$ represents a wavefunction with a hole in spin-orbital \textit{i} and a particle in the spin-orbital \textit{a}, and c are the CI coefficients. Full CI, which includes all possible excited determinants, is exact within the basis set but computationally prohibitive for all but the simplest systems. Truncated CI methods, such as CIS (singles) and CISD (singles and doubles), are more practical but lack size extensivity.

\subsection{Coupled Cluster Theory} \label{sec:CCTheory}
The coupled cluster (CC) theory is considered the gold-standard method in quantum chemistry. Similarly to CI, the CC method expands the wavefunction as a liinear combination of Slater determinats. However, the CC method results into a size-extensive and size-consistent wavefunction by using an exponential ansatz.
\begin{equation}\label{CCWavefunc}
    | \Psi_{CC} \rangle = e^{\hat{T}} | \Psi_{0} \rangle
\end{equation}
\\Where $\hat{T}$ is the cluster operator, which is the central component of CC theory and is defined as a sum of excitation operators:
\begin{equation}
    \hat{T} = \hat{T}_1 + \hat{T}_2 + \hat{T}_3 + \dots + \hat{T}_N
\end{equation}
where $N$ is the total number of electrons in the system. Each term in this sum corresponds to a specific level of excitation:
\begin{itemize}
    \item $\hat{T}_1 = \sum_{i}^{\text{occ}} \sum_{a}^{\text{virt}} t_i^a a_a^{\dagger} a_i$ represents single excitations.
    \item $\hat{T}_2 = \frac{1}{4} \sum_{i,j}^{\text{occ}} \sum_{a,b}^{\text{virt}} t_{ij}^{ab} a_a^{\dagger} a_b^{\dagger} a_j a_i$ represents double excitations.
    \item Higher-order excitation operators $\hat{T}_3, \hat{T}_4, \dots$ describe simultaneous excitation of three, four, and more electrons.
\end{itemize}
The coefficients $t_i^a$, $t_{ij}^{ab}$, etc., are cluster amplitudes to determined by solving the coupled cluster Schr\"{o}dinger equation. The energy of the system is obtained by projecting the Schrödinger equation onto the Hartree-Fock reference determinant:
\iffalse One of the most significant advantages of coupled cluster theory is its property of size consistency. A size-consistent method correctly describes the energy of a system composed of multiple non-interacting subsystems as the sum of the energies of the individual subsystems. The exponential form, when expanded as a Taylor series,
\begin{equation}
    e^{\hat{T}} = 1 + \hat{T} + \frac{1}{2!} \hat{T}^2 + \dots
\end{equation}
inherently includes terms that represent disconnected clusters, which are essential for size consistency. \fi
\begin{equation}\label{eq:CCEnergy}
    E_{CC}=\langle \Psi_{0} | e^{-\hat{T}} \hat{H} e^{\hat{T}} | \Psi_{0} \rangle
\end{equation}
Using the Baker-Campbell-Hausdorff (BCH) expansion, the exponential operators in Eq. \ref{eq:CCEnergy} can be simplified to a series of commutators which ends at the fourth order. The cluster operator $\hat{T}$ can be truncated at different levels of excitation:
\begin{itemize}
    \item \textbf{CCD} (Coupled Cluster Doubles): This is the simplest approximation in the CC family, where the cluster operator is truncated to include only single excitations: $\hat{T} \approx \hat{T}_2$. Due to the Brilluin's theorem, the amplitudes of single excitations are 0. 
    \item \textbf{CCSD} (Coupled Cluster Singles and Doubles): This is one of the most widely used and generally accurate coupled cluster methods, where the cluster operator includes both single and double excitations:$\hat{T} \approx \hat{T}_1 + \hat{T}_2$.
    \item \textbf{CCSDT} (Coupled Cluster Singles, Doubles, and Triples): $\hat{T} \approx \hat{T}_1 + \hat{T}_2 + \hat{T}_3$.
    The hierarchy can be extended to include even higher levels of excitation, converging to the Full Configuration Interaction (Full CI) limit. Full CI includes all possible excitations within a given one-electron basis set and represents the exact solution to the non-relativistic Schrödinger equation in that basis.
\end{itemize}
\begin{table}[h!]
    \centering
    \begin{tabular}{ccc}
        Method & Operation count & Memory \\
        \hline
        HF & $O(N^4)$ & $O(N^4)$ \\
        MP2 & $O(N^5)$ & $O(N^4)$ \\
        CCD/CCSD & $O(N^6)$ & $O(N^4)$ \\
        CCSDT & $O(N^8)$ & $O(N^6)$ \\
        CC2 & $O(N^{5})$ & $O(N^4)$ \\
    \end{tabular}
    \caption{Computational Scaling of quantum chemistry Methods.}
    \label{tab:qc_scaling}
\end{table}

\subsection{Second Approximate Coupled Cluster}\label{sec:CC2Theory}
Second Approximate Coupled Cluster (CC2) belongs to the broader family of CCn approximate coupled cluster methods, where the 'n' in CCn indicates the level of approximation within a perturbative hierarchy. These methods aim to reduce the computational cost associated with standard CC truncations while still retaining a reasonable level of accuracy.\\

In CC2 the equations for the single amplitudes ($t^a_i$) are the same as those in CCSD, while the doubles amplitudes ($t^ab_ij$) are calculated using the non-iterative expression for MP2 (Eq \ref{eq:MP2}). The resulting expression for the CC2 correlation energy is:
\begin{equation}\label{CC2Energy}
    E_{CC2} = \sum_{ij}^{occ} \sum_{ab}^{virt} \frac{1}{4}\frac{|\langle i j || a b \rangle|^2}{\epsilon_a + \epsilon_b - \epsilon_i - \epsilon_j}  + \sum_{i}^{occ} \sum_{a}^{virt} \hat{F}_{ai} t^a_i 
\end{equation}

The perturbative treatment of the doubles amplitudes in CC2, reduces the computational cost compared to CCSD, Table \ref{tab:qc_scaling}. While this approximation can lead to a less accurate description of electron correlation, the inclusion of zeroth-order singles amplitudes allows for an approximate description of orbital relaxation, which often leads to higher quality results compared to MP2.

\section{Equation-of-Motion Methods} \label{sec:eom_theory}
Equation-of-Motion Coupled Cluster (EOM-CC) methods are an extension of ground-state coupled cluster theory which provide a framework for calculating a variety of excited (EE), ionized (IP) and electron-attached (EA) states. In the EOM-CC, the target electronic state is described by applying a linear excitation operator $\hat{R}$ to a reference state, which typically is the coupled cluster wavefunction of the ground state. The target state wavefunction can then be expressed as $|\Psi_{\text{EOM}}\rangle = \hat{R} |\Psi_{\text{CC}}\rangle = \hat{R} e^{\hat{T}} |\Phi_{\text{HF}}\rangle$.\\
The form of the operator $\hat{R}$ is similar to the cluster operator and chosen to access the desired target state. In the case of EOM-EA, the electron attachment operator $R_{EA}$ includes terms that describe the addition of one electron to an unoccupied orbital (a one-particle creation operator), terms that describe the addition of one electron accompanied by the excitation of another electron from an occupied to an unoccupied orbital (a two-particle and one-hole creation operator), and so on:
\begin{equation}\label{eq:R_EA}
    \hat{R}_{EA} = \hat{R}_{1_{EA}} + \hat{R}_{2_{EA}} + \ldots = \sum_{a} r^a a_a^{\dagger} + \frac{1}{2}\sum_{ab} \sum_{i} r^{ba}_{i} a_b^{\dagger} a_a^{\dagger} a_i + \ldots
\end{equation}
\begin{figure}
    \centering
    \medskip
    \includegraphics[width=.7\textwidth]{EOM_EA}
    \caption{EOM-EA.}
    \label{fig:EOM}
  \end{figure}

  Where $a$ and $b$ denote virtual orbitals, $i$ denotes an occupied orbital, and $r^a$ and $r^{ba}_{i}$ are the coefficients (amplitudes) to be determined. The electron affinities (EAs) of the system are then obtained as the eigenvalues of the similarity-transformed Hamiltonian:
\begin{equation}
    \bar{H}_{N} \bar{R} | \Psi_0 \rangle = \Delta E_{EOM} \bar{R} | \Psi_0 \rangle
\end{equation}
\begin{equation}
    \bar{H}_{N} = e^{-T} H e^{T} - \langle \Psi_0 | e^{-T} H e^{T} | \Psi_0 \rangle
\end{equation}

\subsection{Dyson orbitals}

Dyson orbitals are defined as the overlap between the wavefunction of an initial $N$-electron state ($|\Psi_0^N\rangle$) and the wavefunction of the final state with $N\pm1$ electrons ($|\Psi_f^{N\pm1}\rangle$).
\begin{equation}\label{eq:dyson}
    \phi_{d} = \langle \Psi_0^N | \Psi_f^{N\pm1} \rangle = \langle \Psi_0^N | \hat{R}\,\Psi_0^{N} \rangle 
\end{equation}

Because the the terms differ in one electron, the result of the overlap is a vector instead of a scalar, and can be expressed as a linear combination of the molecular orbitals ($\phi_p(r)$) of the reference wavefunction:

\begin{equation}
    \phi_{Dyson}(r) = \sum_p \gamma_p \phi_p(r)
\end{equation}

where $\gamma_p$ are the coefficients that quantify the contribution of each molecular orbital to the Dyson orbital.\\

Physically, Dyson orbitals can be interpreted as representing the correlated state of the electron that is either removed or added to it. They can be used for the interpretation and prediction of photoelectron spectra as they contain all the information required to calculate diffrential corss-sections, $\frac{d\sigma}{d\Omega_k}$:

\begin{equation}
    \frac{d\sigma}{d\Omega_k} = \frac{4\pi^2kE}{c}|\langle \phi_d | \mu | \Psi^{el}_k \rangle |^2
\end{equation}

Where where \textit{k} is the magnitude of the photoelectron wavevector, \textit{E} is the energy of the ionizing radiation, and \textit{c} is the speed of light, $\mu$ is the dipole operator, and $\Psi^{el}_k$ is the photoelectron wavefunction. 

%%%%%%%%%%%%%%%%%%%%%%%%%%%%%%%%%%%%%%%%%%%%%%%%%%
% Keep the following \cleardoublepage at the end of this file, 
% otherwise \includeonly includes empty pages.
\cleardoublepage

% vim: tw=70 nocindent expandtab foldmethod=marker foldmarker={{{}{,}{}}}