% !TeX root = ../../thesis.tex
\chapter*{Abstract}                                 \label{ch:abstract}
\addcontentsline{toc}{chapter}{Abstract}


This thesis investigates the two anionic states of ubiquinone (CoQ), a valence-bound (VBS) and non-valence dipole-bound states (DBSs). DBSs, where an excess electron is weakly bound by the molecular dipole, have sparked interest as potential `doorway' states in electron transfer processes, yet their study is challenging due to their diffuse nature and sensitivity to electron correlation.

The primary theoretical framework employed is the equation-of-motion coupled-cluster (EOM-CC) theory, using the cost-effective second-order approximation, CC2. First, the EA-EOM-CC2 method for calculating electron affinities of both VBSs and DBSs anions is benchmarked and validated. Dyson orbitals for EOM-CC2 are implemented, providing a robust tool for characterizing electron attachment/detachment processes and calculating properties like photodetachment cross-sections.

Investigations on CoQ analogues, Q0 (2,3-dimethoxy-5-methyl-1,4-benzoquinone) and Q1 (CoQ with one isoprene unit), reveal a strong interplay between molecular conformation and the stability of DBSs, mediated by changes in the molecular dipole moment. The study also explored the influence of the local molecular environment on these anionic states using small cluster models.

Results demonstrate that EA-EOM-CC2, with appropriate basis sets, effectively describes both VBSs and DBSs. The newly implemented EOM-CC2 Dyson orbitals offer a reliable and computationally efficient alternative to higher-order methods. This research enhances the understanding of how structural and environmental factors govern the properties of anionic states in biologically relevant quinones, providing insights and tools for future studies in more complex systems.

%%%%%%%%%%%%%%%%%%%%%%%%%%%%%%%%%%%%%%%%%%%%%%%%%%
% Keep the following \cleardoublepage at the end of this file, 
% otherwise \includeonly includes empty pages.
\cleardoublepage

% vim: tw=70 nocindent expandtab foldmethod=marker foldmarker={{{}{,}{}}}
