% !TeX root = ../../thesis.tex
\chapter*{Abstract}                                 \label{ch:abstract}
\addcontentsline{toc}{chapter}{Abstract}


The redox reactions of Ubiquinone (CoQ) are a key step in cellular respiration. CoQ, is able to support two anionic states; a valence-bound (VBS) and non-valence dipole-bound state (DBS). In DBSs the excess electron is weakly bound by the molecular dipole, and have sparked interest as potential `doorways' for electron transfer processes. However, their theoretical study is challenging due to the diffuse nature of the electron cloud and sensitivity to electron correlation.

In this work, the cost-effective electron-attachment equation-of-motion second-order approximated coupled-cluster method (EA-EOM-CC2) is used to study the anionic states of CoQ. Firstly, EA-EOM-CC2 is benchmarked and validated for calculating electron affinities of both VBSs and DBSs anions. Moreover, Dyson orbitals for EOM-CC2 are implemented, providing a tool for characterizing electron attachment/detachment processes and calculating properties like photodetachment cross-sections.

CoQ analogues, Q\textsubscript{0} (2,3-dimethoxy-5-methyl-\textit{p}-benzoquinone) and Q\textsubscript{1} (Q\textsubscript{0} with one isoprene unit at position 6), are investigated, revealing a strong interplay between the conformation of the methoxy chains and the stability of the DBS, mediated by changes in the molecular dipole moment. It is shown that the relative orientation of the dipole is as important as its strength to the electron binding energy. The polarity of the local molecular environment on these anionic states is shown to have a strong effect on both anionic states. This research enhances the understanding of how structural and solvent factors govern the electron transfer processes in biologically relevant quinones, providing insights on how the local protein environment could favour or disfavour the existence of non-valence states.

%%%%%%%%%%%%%%%%%%%%%%%%%%%%%%%%%%%%%%%%%%%%%%%%%%
% Keep the following \cleardoublepage at the end of this file, 
% otherwise \includeonly includes empty pages.
\cleardoublepage

% vim: tw=70 nocindent expandtab foldmethod=marker foldmarker={{{}{,}{}}}
