% !TeX root = ../../thesis.tex
\chapter*{Beknopte samenvatting}
\addcontentsline{toc}{chapter}{Beknopte samenvatting}

% !LTeX spellcheck = nl_NL

\textbf{GET AN INDIGENOUS SPCIMEN TO CHCK THIS TRANSLATION}\\
Deze thesis onderzoekt de twee anionische toestanden van ubiquinon (CoQ): een valentiegebonden toestand (VBS) en een niet-valentie dipoolgebonden toestand (DBS). DBS'en, waarbij een extra elektron zwak gebonden wordt door het moleculaire dipoolmoment, hebben veel interesse opgewekt als mogelijke `doorgangstoestanden' in elektronentransferprocessen. Het bestuderen van deze toestanden is echter uitdagend door hun diffuse aard en gevoeligheid voor elektronencorrelatie.

Het theoretisch kader dat hiervoor gebruikt wordt, is de `equation-of-motion coupled-cluster' (EOM-CC) theorie, toegepast in de kostenefficiënte tweede-orde benadering, CC2. Eerst wordt de EA-EOM-CC2 methode om elektronenaffiniteiten van zowel VBS- als DBS-anionen te berekenen, gevalideerd en gebenchmarkt. Dyson-orbitalen voor EOM-CC2 zijn geïmplementeerd, wat een krachtig hulpmiddel biedt om elektronenaanhechtings- en losmaakprocessen te karakteriseren en eigenschappen zoals fotodetacheringdoorsneden te berekenen.

Onderzoek naar CoQ-analogen, Q0 (2,3-dimethoxy-5-methyl-1,4-benzoquinon) en Q1 (CoQ met één isopreeneenheid), toont een sterk samenspel tussen de moleculaire conformatie en de stabiliteit van DBS'en, gestuurd door veranderingen in het moleculaire dipoolmoment. Ook werd de invloed van de lokale moleculaire omgeving op deze anionische toestanden onderzocht met behulp van kleine clustermodellen.

De resultaten tonen aan dat EA-EOM-CC2, met geschikte basissets, zowel VBS- als DBS-toestanden accuraat beschrijft. De nieuw ge{\"i}mplementeerde EOM-CC2 Dyson-orbitalen bieden een betrouwbaar en rekenefficiënt alternatief voor duurdere methodes. Dit onderzoek levert nieuwe inzichten op in hoe structurele en omgevingsfactoren de eigenschappen van anionische toestanden in biologisch relevante chinonen beïnvloeden, en biedt hulpmiddelen voor toekomstige studies in complexere systemen.

%%%%%%%%%%%%%%%%%%%%%%%%%%%%%%%%%%%%%%%%%%%%%%%%%%
% Keep the following \cleardoublepage at the end of this file, 
% otherwise \includeonly includes empty pages.
\cleardoublepage

% vim: tw=70 nocindent expandtab foldmethod=marker foldmarker={{{}{,}{}}}
