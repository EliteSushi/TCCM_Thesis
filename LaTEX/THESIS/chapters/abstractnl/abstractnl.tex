% !TeX root = ../../thesis.tex
\chapter*{Beknopte samenvatting}
\addcontentsline{toc}{chapter}{Beknopte samenvatting}

% !LTeX spellcheck = nl_NL

De redoxreacties van ubiquinon (CoQ) vormen een essentiële stap in de cellulaire respiratie. CoQ is in staat om twee anionische toestanden te ondersteunen: een valentiegebonden toestand (VBS) en een niet-valentiegebonden dipoolgebonden toestand (DBS). In DBS’en wordt het overtollige elektron zwak gebonden door de moleculaire dipool, wat deze toestanden bijzonder interessant maakt als mogelijke ‘toegangspoorten’ voor elektronentransferprocessen. Hun theoretische studie blijft echter een uitdaging, vanwege de diffuse aard van de elektronenwolk en de gevoeligheid voor elektronencorrelatie.

In deze studie wordt de kostenefficiënte `electron-attachment equation-of-motion second-order approximate coupled-cluster' methode (EA-EOM-CC2) toegepast om de anionische toestanden van CoQ te onderzoeken. Eerst wordt EA-EOM-CC2 gebenchmarkt en gevalideerd voor het berekenen van de elektronaffiniteiten van zowel VBS- als DBS-anionen. Daarnaast worden Dyson-orbitalen geïmplementeerd binnen het EOM-CC2-formalisme, wat een waardevol instrument biedt voor het karakteriseren van elektron aanhechtings en verwijderings processen, evenals voor het berekenen van eigenschappen zoals elektron photoverwijderings doorsnedes.

Twee CoQ-analogen worden onderzocht: Q\textsubscript{0} (2,3-dimethoxy-5-methyl-\textit{p}-benzoquinon) en Q\textsubscript{1} (Q\textsubscript{0} met één isopreeneenheid op positie 6). De resultaten tonen een sterke wisselwerking aan tussen de conformatie van de methoxyketens en de stabiliteit van de DBS, gemedieerd door veranderingen in het moleculaire dipoolmoment. Er wordt aangetoond dat de relatieve oriëntatie van de dipool minstens even bepalend is als de sterkte ervan voor de bindingsenergie van het elektron. Tevens blijkt de polariteit van de lokale moleculaire omgeving een significante invloed uit te oefenen op beide anionische toestanden. Deze bevindingen dragen bij tot een beter begrip van structurele factoren en solventeffecten die elektronentransfer in biologisch relevante quinonen sturen, en bieden inzicht in hoe de lokale eiwitomgeving het bestaan van niet-valentiegebonden toestanden kan bevorderen of verhinderen.

%%%%%%%%%%%%%%%%%%%%%%%%%%%%%%%%%%%%%%%%%%%%%%%%%%
% Keep the following \cleardoublepage at the end of this file, 
% otherwise \includeonly includes empty pages.
\cleardoublepage

% vim: tw=70 nocindent expandtab foldmethod=marker foldmarker={{{}{,}{}}}
