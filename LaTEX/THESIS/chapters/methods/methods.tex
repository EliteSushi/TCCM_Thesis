% !TeX root = ../../thesis.tex
\chapter{Computational Methods}\label{ch:methods}

\textbf{TODO: complete and divide in subsection}

All electronic structure calculations were performed using the developer's copy of the \textit{Q-Chem} software \cite{QChem5}. In all computations the frozen-core
approximation is used, only the valence electrons are correlated, as well as the resolution of the identity (RI) approximation, auxiliary basis functions are used to approximate the two-electron integrals\cite{hattig2000cc2}.

For the EOM-EA calculations, the reference wavefunction was obtained as the restricted Hartree-Fock (RHF) solution of the ground state of the neutral molecule. Unless explicitly mentioned, calculations were performed at using the aug-cc-pVDZ basis set \cite{dunning1989gaussian} further augmented by 3 s-shells on hydrogen atoms and 6 s- and 3 p-shells on all non-hydrogen atoms \cite{paran2024performance} to properly model the non-valence states. The coefficients of the extra functions were obtained by successively halving the most diffuse function of the original set.\label{sec:methods:basis}.

CC2 Dyson orbitals for EOM variants described in section \ref{sec:theory_dyson} and appendix \ref{ch:appendix:dyson} were implemented as described, and will be released in an upcoming verion of \textit{Q-Chem}.

All closed-shell quinone model geometries were optimized using the TPSS functional\cite{tao2003climbing} with Grimme's pair-wise dispersion corrections with Becke-Johnson damping (D3BJ)\cite{grimme2011effect}, and the minimally augmented\cite{zheng2011minimally} def2-TZVP basis sets\cite{weigend2005balanced} (ma-def2-TZVP), following the work in \cite{schulz2018systematic}. For the scan calculations, each single point was optimized constraining its relevant angles by the method of Lagrange multipliers; dihedrals of the methoxy chains of Q0 and Q1,and the isoprene tail of Q1.
In the case of quinone + aminoacid models, crystal structures were taken from the Protein Data Bank (PDB). Hydrogens were added using \textit{PyMOL}'s \cite{PyMOL} \texttt{add\_H}  functionality, and relaxed using the method above (fixing the rest of the heavy atoms).

For the scans of quinone + molecule, each subsystem was independently optimized and put together with any further refinement.

For quinone systems, only EOM-EA right Dyson orbitals were computed to speed up the calculations by avoiding the need to compute the lambda terms.

Photoionization and Photodetachment cross-sections were calculated using the \textit{ezDyson} package \cite{gozem2022ezspectra,gozem2015photoionization}.






%%%%%%%%%%%%%%%%%%%%%%%%%%%%%%%%%%%%%%%%%%%%%%%%%%
% Keep the following \cleardoublepage at the end of this file, 
% otherwise \includeonly includes empty pages.
\cleardoublepage


% vim: tw=70 nocindent expandtab foldmethod=marker foldmarker={{{}{,}{}}}