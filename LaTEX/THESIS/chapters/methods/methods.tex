% !TeX root = ../../thesis.tex
\chapter{Computational Details}\label{ch:methods}

All files pertinent to this research are accessible via the \href{https://github.com/EliteSushi/TCCM_Thesis}{GitHub repository}. Electronic structure calculations were conducted using a developer's version of the \textit{Q-Chem} software package \cite{QChem5}. The frozen-core approximation was employed in all calculations, meaning only valence electrons were correlated. Furthermore, the resolution of the identity (RI) approximation was utilised throughout \cite{hattig2000cc2}.\\

\section{Performance Evaluation of EOM-CC2 Methods}

\subsection{Basis Set Dependence of EA-EOM-CC2 for Dipole-Bound Anions} \label{sec:methods:basis}

The influence of the basis set on EA-EOM-CC2 calculations was investigated using a dataset of 14 molecules exhibiting radical dipole-bound anions, as detailed in Ref. \citenum{paran2024performance}. These molecules were chosen to represent a broad spectrum of dipole moments and polarisabilities. The calculations employed modified aug-cc-pV\textit{X}Z basis sets \cite{dunning1989gaussian}, with the cardinality \textit{X} ranging from double to quadruple. To accurately describe non-valence states, supplementary diffuse functions were incorporated. A basis set denoted as aug-ccpV\textit{X}Z+\textit{(2n)}s\textit{(n)}p signifies the addition of \textit{2n} s-type and \textit{n} p-type shells to heavy atoms, and \textit{n} of each to hydrogen atoms. The exponents for these extra functions were generated by systematically halving the exponent of the most diffuse s and p functions in an even-tempered fashion. \\

\subsection{Assessment of EA-EOM-CC2 for Valence-Bound Radical Anion States in Quinones}

The efficacy of EA-EOM-CC2 for treating valence-bound radical anion states of quinones was evaluated using a set of 10 molecules previously established in Ref. \citenum{schulz2018systematic}.
The molecular geometries corresponded to the optimised structures of the neutral species, which the original authors had optimised at the TPSS/ma-def2-TZVP level of theory. For the spin-component scaling (SCS) points, the factors applied were (c\textsubscript{ss}=1/3) for same-spin and (c\textsubscript{os}=6/5) for opposite-spin contributions, following previous works\cite{grimme2003improved,paran2024performance,shaalan2022accurate}.\\

\subsection{Photoelectron Cross-sections from EOM-CC2 and EOM-CCSD}

Dyson orbitals for the EOM-CC2 variants, as outlined in Section \ref{sec:theory_dyson} and appendix \ref{ch:appendix:dyson}, were implemented as described. These will be made available in a forthcoming release of \textit{Q-Chem}.

To ascertain the quality of the EOM-CC2 Dyson orbitals, photoelectron cross-sections were computed using the \textit{ezDyson} software \cite{gozem2022ezspectra,gozem2015photoionization}. This analysis involved a set of 20 transitions, which were then compared against results obtained from EOM-CCSD Dyson orbitals and the predominant HF orbital within the EOM-CCSD Dyson orbital. The test set comprised the standard \textit{ezDyson} benchmark set from Ref. \citenum{gozem2015photoionization}, augmented with 4 photodetachment transitions from dipole-bound anions and 7 photodetachment transitions from valence-bound anions. Graphical representations of these results are provided in appendix \ref{ch:appendix:crosssection}.\\

\section{Ubiquinone Computational Models}

\subsection{Potential Energy Surfaces of Quinone Models}

A systematic grid scan of the dihedral angles of the methoxy chains in Q\textsubscript{0} and Q\textsubscript{1} was performed in 20\degree{} increments. At each grid point, the molecular structure was optimised while constraining the relevant dihedral angles of the methoxy chains (and the isoprene tail for Q\textsubscript{1}) using the method of Lagrange multipliers. The angle and dihedral of the isoprene tail relative to the benzene ring were maintained at 114.1\degree{} and 111.8\degree{}, respectively, values derived from the crystal structure of Complex I (PDB: 6i0d) \cite{gutierrez2020key}. All optimised structures are available in the \href{https://github.com/EliteSushi/TCCM_Thesis}{GitHub repository}. Geometry optimisations utilised the TPSS functional \cite{tao2003climbing} combined with Grimme's D3BJ pairwise dispersion correction (employing Becke-Johnson damping) \cite{grimme2011effect}, and the minimally augmented ma-def2-TZVP basis set \cite{zheng2011minimally,weigend2005balanced}. This level of theory has previously demonstrated good performance for quinone systems \cite{schulz2018systematic}.

The resultant geometries served as the basis for subsequent EOM-EA calculations. The reference wavefunction for these calculations was the RHF solution for the ground state of the neutral molecule. Unless otherwise specified, these calculations were performed with the aug-cc-pVDZ+6s3p basis set (\textit{vide supra}) to ensure an adequate description of non-valence states.

For the Dyson orbital computations, only right Dyson orbitals were calculated. This approach expedited the calculations by obviating the need to determine the lambda amplitudes.

%Hydrogens were added using \textit{PyMOL}'s \cite{PyMOL} \texttt{add\_H}  functionality, and relaxed using the method above (fixing the rest of the heavy atoms).\\

\subsection{Interaction Scans with Small Molecules}

Cluster models were constructed based on the Q\textsubscript{0} structure with dihedral angles $\mathrm{\psi=160\degree{}}$ and $\mathrm{\phi=-80\degree{}}$. For interactions with dipolar molecules (all second-row hydrides), the hydride was positioned along the dipole moment vector of the quinone, originating from the Q\textsubscript{0} centre of mass. The dipole moments of the quinone and hydride were then aligned in either parallel or antiparallel orientations. The interaction distance was defined as the separation between the centre of mass of the quinone and the heavy atom of the hydride. In the case of the noble gas cage, the centre of mass of Q\textsubscript{0} was coincident with the centre of the icosahedron. The geometries of these generated systems were not subjected to further optimisation. All geometries, along with the Python scripts used for their generation, are available in the \href{https://github.com/EliteSushi/TCCM_Thesis}{GitHub repository}.\\

%%%%%%%%%%%%%%%%%%%%%%%%%%%%%%%%%%%%%%%%%%%%%%%%%%
% Keep the following \cleardoublepage at the end of this file,
% otherwise \includeonly includes empty pages.
\cleardoublepage

% vim: tw=70 nocindent expandtab foldmethod=marker foldmarker={{{}{,}{}}}