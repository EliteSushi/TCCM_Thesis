% !TeX root = ../../thesis.tex
\chapter{Conclusion and Outlook}\label{ch:conclusion}
This thesis presents a theoretical investigation of anionic states in ubiquinone derivatives, a key molecule in biological electron transfer which supports both a dipole-bound state and valence-bound states, using EA-EOM-CC2 techniques.\\

Analysis of basis set dependence for EA-EOM-CC2 calculations for DBS electron binding energies revealed that, although larger cardinality sets (e.g.\ aug-cc-pVTZ) are beneficial, the inclusion sufficiently diffuse functions is critical for describing the spatial extent of DBS orbitals. In general CC2 tends to slightly overbind these states when compared to CCSD. 
The VBSs electron binding energies are consistently overbound by EA-EOM-CC2. The diffuse functions integral to DBSs have minimal impact on VBS energies, as the latter extra electron density is much more localized in space. For the same reason, the effect cardinality of the used basis set is notably larger than in the case of DBS. Inclusion of SCS notably improves the EA-EOM-CC2 results, however unscaled CC2 shows a consistent error and is able to recover trends.
EOM-CC2 Dyson orbitals were implemented within the \texttt{ccman2} module of \textit{QChem}. They were validated as a resource-efficient tool by comparing them to EOM-CCSD Dyson orbitals and Koopmans' approximation in the calculation of photodetachment cross-sections.\\

The quinone system was investigated using two models: Q\textsubscript{0}, corresponding to the bare quinone moiety of ubiquinone, and Q\textsubscript{1}, which includes an isoprene tail. In both cases, the surfaces defined by the rotation of the methoxy chains were mapped for both the conformational ground electronic energy, dipole moment strength, and VBS and DBS electron binding energies. 
For both molecules the dipole strength of the quinone is strongly determined by the orientation of the methoxy groups, which in turn affects the VBS and DBS energies. The VBS binding energy varies about 25\% across the conformational space, depending on the eletron withdrawing (stabilizing) or donating (destabilizing) character of the methoxy groups. \\
The DBS binding energy is much more sensitive to the methoxy chain conformations, becoming unbound when the dipole is not strong enough. In both cases, different regions of strong dipole moment can be identified. Remarkably, each region can be mapped to a distinct DBS population regarding the correlation between the dipole moment and the DBS binding energy. In Q\textsubscript{0}, the two regions correspond to the dipole moment pointing roughly perpendicular or parallel to the plane defined by the benzene ring. Within each region, the DBS electronic density interacts with the rest of the electronic density differently, and the binding energy and the dipole strength show a nearly linear relation.\\
Extending to Q\textsubscript{1}, the fixed isoprene tail breaks the symmetry of Q\textsubscript{0}. For structures with the dipole parallel to the benzene ring, the DBS remains largely unaltered. For structures with the dipole perpendicular to the benzene ring, the isoprene tail stabilises the DBS if both the tail and the DBS are on the same side of the benzene, and vice versa. This is rationalized by the constructive or destructive alignment of the dipole moment of the isoprene tail with the dipole moment created by the methoxy chains, and by the extra stabilizing dispersion interactions between the DBS and the isoprene electronic density. The VBS binding energy is not particularly sensitive to the isoprene tail's presence, as it is dominated by the benzene ring's electronic density. It is emphasized that only one configuration of the isoprene tail was considered; future work extend the understanding of the effect of both different conformations and addition of more units.\\

The interaction between Q\textsubscript{0} and the solvent was investigated by positioning a small molecule along the dipole moment vector of the quinone. This setup involved two configurations: one where the dipoles were oriented in parallel and another where they were antiparallel. The distance between the molecules was then varied. The results underscored the sensitivity of both VBSs and DBSs to the local environment. This effect was found to be considerably larger than the configuration of the methoxy chains, which contrasts with previous works suggesting that the electron affinity of ubiquinone is largely controlled by the methoxy groups\cite{schulz2018systematic,taguchi2013tuning,taguchi2013conformational,deAlmeida2014effect}. It was found that the DBS can be stabilized by an order of magnitude by a single molecule.
Though this provides qualitative understanding of the system, explicit solvation and bigger and more realistic cluster models could quantitatively characterise environmental influence on VBS and DBS formation in protein settings. Other techniques that could be utilised are electrostatic embedding.\\

Furthermore, for larger systems that include more solvating molecules, introducing molecular mechanics (MM) or QM/MM simulations to account for structural fluctuations would offer insights into configurations where these states could be relevant. Then, interconversion rates could be estimated between the VBS and the DBS, and between them and a potential electron donor. This would provide a more complete picture of the ubiquinone anion states in a biological context.\\

Overall, this thesis advances the theoretical understanding of non-valence anionic states in biological molecules and the computational techniques applied to them. It offers a thorough computational analysis of non-valence anions, supplies methodological insights, and delivers a closer characterisation of ubiquinone anion states. These outcomes pave the way for further studies of the complex physics and chemistry associated with such species in biological frameworks.\\
Finally, from a pedagogical perspective, this work comprehensively spans multiple facets of theoretical chemistry and computational modelling. From a derivation of an algebraic expression for EOM-CC2 Dyson orbitals, to their implementation in quantum-chemistry software, to their application to a sizeable \textit{(bio)}chemical problem.\\

%%%%%%%%%%%%%%%%%%%%%%%%%%%%%%%%%%%%%%%%%%%%%%%%%%
% Keep the following \cleardoublepage at the end of this file, 
% otherwise \includeonly includes empty pages.
\cleardoublepage

% vim: tw=70 nocindent expandtab foldmethod=marker foldmarker={{{}{,}{}}}
